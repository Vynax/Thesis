\chapter*{摘~~要}
\addcontentsline{toc}{chapter}{中文摘要}

%基本資訊

\noindent
論文名稱:Open Channel SSD? \\%Robot Framework測試腳本重構工具的改善:增加重構方法之多元選擇\\
頁數:\\
校所別:臺北科技大學~資訊工程系碩士班\\
畢業時間:一百一十學年度第二學期\\
學位:碩士\\
研究生:吳承岳\\
指導教授:陳碩漢教授\\
%\hspace*{\fill}\\
\noindent
關鍵字:LightNVM、Open Channle SSD、SSD、Q-Learning、強化學習、寫入效能、Linux、Kernel Module\\
%\hspace*{\fill}\\
%
\indent
針對原本在 Linux 核心中既有的 SSD管理系統模組 LightNVM,本論文加入一種機器學習演算法,除了保有原先所提供的管理Open Channle SSD之功能,包含管理讀取、寫入、Garbage Collection等功能,我們在寫入之中加入將資料依據特性分類,將經常更改的熱門資料 (Hot data)放在一起,以及將經過許久才修改一次的冷門資料(Cold Data)擺在一起,讓這個模組在做 Garbage Collection 時可以容易將資料一併整理。

\indent
而這個演算法是一種機器學習演算法 Q-Learning,運用其特殊的結構 Q-Table,將資料劃分等級,紀錄每次寫入時獲得的獎勵 (reward);每次要寫入時,就會用之前紀錄的獎勵值來推測寫在哪個位置的狀況會比較好,以決定當下寫入的位置。讓 LightNVM 所管理的 Open Channle SSD 不再只能循序寫入,現在可以透過集中 Hot Data 以及 Cold Data,來提升整體效率。
%針對本實驗室開發的一個Robot Framework測試腳本重構工具RF Refactoring,本論文提出兩種功能延伸,除了保有原先所提供的三種重構功能,重新命名關鍵字、重新命名變數及修改關鍵字介面之外,新增了抽取重複步驟成為新關鍵字、移動關鍵字宣告的功能,其能夠使工具更加完備,讓開發人員進行相關重構時,不再只能利用搜尋取代的方式進行重構,且不需進行不必要的人工檢查,進而提升重構效率,而在重構方法的選擇上也更加多元。


%\indent
%Robot Framework是一種利用關鍵字驅動的自動化驗收測試框架,其擁有良好的可讀性以及擴充性,因此許多專案都會使用Robot Framework開發自動化驗收測試。當專案到達一定的規模大小,並且團隊有既定的程式碼風格時,測試團隊常會遭遇需要重構關鍵字及測試腳本的問題,例如修改關鍵字名稱、抽取重複步驟等等。過去團隊所使用之重構工具提供了三種重構功能,重新命名關鍵字、重新命名變數、修改關鍵字介面,其幫助測試團隊在進行重構時,不再只能利用搜尋取代的方式進行重構,且不需進行不必要的人工檢查,進而提升重構效率。但目前其重構功能仍然是不足的,測試團隊在進行其他類型的重構時,仍須使用目前用來撰寫Robot Framework測試腳本的整合式開發環境(RED、Visual Studio Code)中的搜尋取代工具進行重構,因而造成重構效率的降低,以及人工檢查的疏忽,導致重構的錯誤和缺漏等等。
%
%\indent
%本論文將為團隊使用中的重構工具增加不同的重構功能,使測試團隊在選擇重構方法時,能夠更加多元,而在重構過程中,也能夠省去多餘的人工檢查,避免搜尋的缺漏及取代的錯誤。