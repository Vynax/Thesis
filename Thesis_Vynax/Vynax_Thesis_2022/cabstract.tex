\chapter*{摘~~要}
\addcontentsline{toc}{chapter}{中文摘要}

%基本資訊

\noindent
論文名稱:以非監督式學習增進開放通道固態硬碟之存取效能 \\%Robot Framework測試腳本重構工具的改善:增加重構方法之多元選擇\\
頁數:\\
校所別:臺北科技大學~資訊工程系碩士班\\
畢業時間:一百一十學年度第二學期\\
學位:碩士\\
研究生:吳承岳\\
指導教授:陳碩漢 博士\\
%\hspace*{\fill}\\
\noindent
關鍵字:LightNVM、Open Channle SSD、SSD、Q-Learning、強化學習、寫入效能、Linux、Kernel Module\\
%\hspace*{\fill}\\
%
\indent
一直以來,SSD都有三個限制,Erase Before Write 跟 Limited P/E Cycles 跟 Asymmertic Program/Erase Unit,而過去都是讓 SSD 之中 的 FTL 獨自管理。但是這樣 SSD 並沒有 Host 的資訊,也就是並沒有跟 Host 溝通,造成 Host 與 SSD 雙方的資訊落差,進而導致效率下降。而 Open Channel SSD 就是為了讓 Host 與 FTL 溝通而改良而成的一種新型 SSD。
%針對原本在 Linux 核心中既有的 SSD 管理系統模組 LightNVM,本論文加入一種機器學習演算法,除了保有原先所提供的管理Open Channle SSD之功能,包含管理讀取、寫入、Garbage Collection等功能,我們在寫入之中加入將資料依據特性分類,將經常更改的熱門資料 (Hot data)放在一起,以及將經過許久才修改一次的冷門資料(Cold Data)擺在一起,讓這個模組在做 Garbage Collection 時可以容易將資料一併整理。

\indent
而本論文針對 Open Channel SSD 提出了利用強化學習來增進效率,並選擇 Q-Learning 作為案例分析,其特殊的結構 Q-Table,可將資料劃分等級,紀錄每次寫入時獲得的獎勵 (reward);每次寫入都會用之前紀錄的獎勵值來推測寫在哪個位置的狀況會比較好,以決定當下寫入的位置。讓 Open Channle SSD 不再只能循序寫入,現在可以透過集中 Hot Data 以及 Cold Data,來提升整體效率。
%而這個演算法是一種機器學習演算法 Q-Learning,運用其特殊的結構 Q-Table,將資料劃分等級,紀錄每次寫入時獲得的獎勵 (reward);每次要寫入時,就會用之前紀錄的獎勵值來推測寫在哪個位置的狀況會比較好,以決定當下寫入的位置。讓 LightNVM 所管理的 Open Channle SSD 不再只能循序寫入,現在可以透過集中 Hot Data 以及 Cold Data,來提升整體效率。


%針對本實驗室開發的一個Robot Framework測試腳本重構工具RF Refactoring,本論文提出兩種功能延伸,除了保有原先所提供的三種重構功能,重新命名關鍵字、重新命名變數及修改關鍵字介面之外,新增了抽取重複步驟成為新關鍵字、移動關鍵字宣告的功能,其能夠使工具更加完備,讓開發人員進行相關重構時,不再只能利用搜尋取代的方式進行重構,且不需進行不必要的人工檢查,進而提升重構效率,而在重構方法的選擇上也更加多元。


%\indent
%Robot Framework是一種利用關鍵字驅動的自動化驗收測試框架,其擁有良好的可讀性以及擴充性,因此許多專案都會使用Robot Framework開發自動化驗收測試。當專案到達一定的規模大小,並且團隊有既定的程式碼風格時,測試團隊常會遭遇需要重構關鍵字及測試腳本的問題,例如修改關鍵字名稱、抽取重複步驟等等。過去團隊所使用之重構工具提供了三種重構功能,重新命名關鍵字、重新命名變數、修改關鍵字介面,其幫助測試團隊在進行重構時,不再只能利用搜尋取代的方式進行重構,且不需進行不必要的人工檢查,進而提升重構效率。但目前其重構功能仍然是不足的,測試團隊在進行其他類型的重構時,仍須使用目前用來撰寫Robot Framework測試腳本的整合式開發環境(RED、Visual Studio Code)中的搜尋取代工具進行重構,因而造成重構效率的降低,以及人工檢查的疏忽,導致重構的錯誤和缺漏等等。
%
%\indent
%本論文將為團隊使用中的重構工具增加不同的重構功能,使測試團隊在選擇重構方法時,能夠更加多元,而在重構過程中,也能夠省去多餘的人工檢查,避免搜尋的缺漏及取代的錯誤。