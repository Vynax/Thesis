\chapter*{誌~謝~}
\addcontentsline{toc}{chapter}{誌謝}

\indent
在此要先感謝我的指導教授陳碩漢老師,由於本身有先離開學校一兩年再回歸學習的環境,雖然大學畢業就是資訊工程系,但是除了為了考試所準備的東西比較熟悉,其他範圍其實已經有點生疏。不過在碩士兩年時光,修了很多作業很多的課程,還有老師的要求不斷的激勵我,讓我在短時間內學習到眾多的專業知識與技能,也讓我有機會鑽研 Linux Kernel,並且包容我的魯莽以及失誤。雖然這段時間有時候也會不知所措,不過最後看到成果,心中還是充滿著喜悅。
\\ \hspace*{\fill} \\
\indent
同時感謝口試委員陳碩漢老師、梁郁珮老師、劉傳銘老師,對於我的論文提出十分多的建議,讓我的論文能夠更加完整。也感謝吳俊青學長和林稟宸學長,在我開始就讀碩士之後,對於我進修課程以及程式設計上給予許多建議。此外也感謝黃國豪同學,在我在寫作業或是研究時如果有想法卡關之類的狀況,都會給予我協助與建議。
此外十分感謝研究室內的同學們、學長們、學弟妹們,在我寫論文的過程中給予我很多幫助,以及兩年內所有的陪伴,在這個環境下成長是一件十分快樂的事情。
\\ \hspace*{\fill} \\
\indent
最後感謝我的家人們對於我在讀研究所時,給予關心以及關懷,讓我可以無後顧之憂準備論文,真的非常感謝你們的支持與鼓勵,讓我知道我不是一個人在面對這一切,並且順利取得碩士學位。
% \indent
%  首先要感謝我的指導教授陳碩漢老師,在這兩年內辛苦的指導,讓我在這些時間內學習到眾多的專業知識與技能,並且讓我有機會擁有一個大型軟體專案開發的經驗,也十分感謝口試委員劉立頌老師、陳碩漢老師,對於我的論文提出十分多的建議,讓我的論文能夠更加完整並且對於研究室內的團隊有所貢獻。

% \indent
%  此外十分感謝研究室內的同學們、學長們、學弟妹們,在我寫論文的過程中給予我很多幫助,並且協助我完成論文所需的實驗,以及兩年內所有的陪伴,有你們在的研究室永遠是最愉快的。

% \indent
%  最後感謝我的家人們對於我在台北讀研究所時,不斷的關心以及關懷,真的非常感謝你們的支持與關心,還有我的女友安安,在我忙碌的時候耐心的陪伴我、鼓勵我,讓我知道我不是一個人在面對這一切,並且順利取得碩士學位。