\chapter*{ABSTRACT}
\addcontentsline{toc}{chapter}{ABSTRACT}

%基本資訊

\noindent
Title:Exploring Hot/Cold Data Separation for Garbage Collection Efficiency Enhancement on Open-Channel SSD\\
Pages:24\\
School: National Taipei University of Technology\\
Department: Computer Science and Information Engineering\\
Time: September,2022\\
Degree: Master\\
Researcher: CHENG-YUEH WU\\
Advisor: SHOU-HAN CHEN, Ph.D.\\
%\hspace*{\fill}\\
Keywords: LightNVM, Open-Channel SSD, SSD, Garbage Collection, Performance, Linux, Kernel Module\\
%\hspace*{\fill}\\
\indent
In the past, the FTL (Flash Translation Layer) in the SSD (Solid State Drive) has been left to manage the SSD internal alone. However, the SSD did not communicate with the Host, resulting in a gap in information between the Host and the SSD, which in turn led to a decrease in efficiency. Open Channel SSD is a new type of SSD that is improved to allow the Host to communicate with the FTL.

\indent
This thesis proposes a method by dividing the data into four different levels, from cold to hot, for Open Channel SSD. We separate the data according to the level to improve the efficiency of Garbage Collection. Open-Channel SSD can no longer only write sequentially. They are able to improve the overall efficiency by centralizing Hot Data and Cold Data now.

%過去都是讓 SSD 之中 的 FTL 獨自管理內部。但是這樣 SSD 並沒有跟 Host 溝通,造成 Host 與 SSD 雙方的資訊落差,進而導致效率下降。而 Open Channel SSD 就是為了讓 Host 與 FTL 溝通而改良而成的一種新型 SSD。 DeepL 翻譯再改的

%This thesis proposes two extensions to RF Refactoring, a Robot Framework test script refactoring tool developed in our laboratory. In addition to the three original features - renaming keyword, renaming variable, and modifying keyword interface - two new features have been added: extracting duplicate steps to a new keyword and relocating keyword definition. With the added features, RF Refactoring is better able to enable developers to carry out relevant refactoring without resorting to search, replace, and manual inspection, thereby improving the efficiency of refactoring.